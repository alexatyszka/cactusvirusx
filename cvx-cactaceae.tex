% -*- program: pdflatex -*-
\documentclass[11pt,letterpaper,titlepage]{article}
\usepackage{evolution}
\usepackage[T1]{fontenc}
\usepackage{tgtermes}
\usepackage{etoolbox}
\usepackage{graphicx}
\hypersetup{
   pdftitle={CVX},
   pdfauthor={Alexa Tyszka, Karolis Ramanauskas, Boris Igi{\'c}}
   }
\usepackage[all]{hypcap}
%\usepackage{doi} % remove url!
%\graphicspath{{figures/}}
\renewcommand{\bibsection}{\section*{References}}

\providecommand{\e}[1]{\ensuremath{\times 10^{#1}}}
\newcommand{\ca}{\textit{ca}.}
\begin{document}
% TITLE =======================================================================
\title{\Large\bf{Genome evolution, taxonomy, and transmission of potexviruses in cacti (\textit{Alphaflexiviridae})}}
\author{Alexa Tyszka$^{1}$, Karolis Ramanauskas$^{1,2}$ and Boris Igi\'{c}$^{1,3,4}$}
\date{
    $^1$Department of Biological Sciences\\
    University of Illinois at Chicago\\
    840 West Taylor St.\ MC067\\
    Chicago, IL 60607, U.S.A.\\
    [\baselineskip]
    $^2$Email: {\tt<kraman2@uic.edu>}\\
    $^3$Email: {\tt<boris@uic.edu>}\\
    $^4$Corresponding author.\\
}
\maketitle

%FIXMEs: 
% Read the log notes (git log) each time!

\begin{linenumbers}
\modulolinenumbers[1]
\setstretch{1.0}

% ABSTRACT ====================================================================
\begin{abstract}

\textit{Potexvirus} species members are positive-sense single-stranded RNA viruses known to infect many flowering plants, including cacti (Cactaceae).
The current viral taxonomic naming schemes in this group often employ informal or outdated host plant names (synonyms), which complicate systematic study.
One such group, often named with a suffix "Virus X," presents a further complication---nearly all of its published sequences are from infections of cultivated plants, in which infections may dramatically affect yield.
Because their host-specificity is broad, the source of infections, the natural distribution of this group, and the significance of infections in wild species of cacti all remain unclear.
The lack of clarity is partly related to low sampling across the Potexviruses that infect cacti. 
And yet, the availability of sampled plant transcriptomes, all of which are practically metatranscriptomes, has recently exploded, along with the decreasing expense and difficulty of conducting RNAseq experiments.
Here, we harness these new tools and perform phylogenetic analyses aimed at clarifying taxonomic diversity, quantifying patterns of tissue expression, diversity, and examining selective pressures across viral genomes. 
The results suggest a novel mode of transmission by sex (pollination) for this viral group, based on significant expression in pollen.
We examine and discuss the implications of our key results for the taxonomy of \textit{Cactaceae}-specific \textit{Potexviruses}, noting their vastly understudied ecological significance.
%B: Maybe note the number of genera or specific taxonomic proposals (if changes are proposed)
%B: the problem here is that these names should basically be retired: %\textit{Cactus Virus X, Zygocactus Virus X, Schlumbergera Virus X, Pitaya Virus X,} and \textit{Opuntia Virus X} are five of the 48 currently described \textit{Potexvirus} species, which all infect valuable ornamental and crop plants, often causing production losses. 
\end{abstract}


\setlength{\emergencystretch}{7.5pt}
\setstretch{1.0}
\setlength{\parindent}{0.25in}

\section*{Introduction}

The plant viriome represents a complex evolutionary interaction between the eukaryotic host and viral vector \citep{delwart_viral_2007}. 
Plant viruses were the earliest characterized viruses, beginning with Mayer's publication on his discovery of \textit{Tobacco Mosaic Virus} in 1886 on tobacco plants \citep{mayer1886mosaikkrankheit} which followed Molisch's 1885 discovery of ``protein bodies'' on cacti: \textit{Schlumbergera truncata} (previously \textit{Epiphylum truncatumin}) \citep{molisch1885merkwurdige}.
Molisch's ``Proteinkörper'' may have been the first true purified viruses.
Perhaps the most recent advancement in virology of the current century has been the advent of faster, cheaper, higher-throughput environmental metagenomic techniques.
The unprecedented data collection has advanced many facets of evolutionary biology \citep{delwart_viral_2007,lefeuvre_evolution_2019,schulz_towards_2017}.
It has become evident through these discoveries that the greenhouse-raised and lab-grown organisms commonly analyzed in experiments represent a small fraction of living diversity. % KR: "living diversity" does not sound right to me.
Metagenomic studies aim to sample hundreds of thousands of genomes and have vastly expanded both the cellular tree of life \citep{schulz_towards_2017,hug_new_2016} and the viral tree of life \citep{gregory_marine_2019,lefeuvre_evolution_2019,shi_redefining_2016}. %check viral tree of life
Viruses display impressive morphological diversity and adaptations, many of which allow them to infect plants \citep{delwart_viral_2007,lefeuvre_evolution_2019}.
Metagenomic analysis of non-model plants has both enriched our genetic knowledge of plants and uncovered novel insights on viral evolution, adaptation, and transmission (citations needed).
A careful study of new methods to investigate the plant viriome provides a view into underlying biological realities that are not currently understood.

The International Committee on Taxonomy of Viruses (ICTV) presently advises the taxonomy and approval of virus nomenclature \citep{simmonds2017virus,lefkowitz2018virus,international2020new}.
The unprecedented amounts of data resulting from metagenomic studies have caused significant revisions in ICTV policy \citep{international2020new,simmonds2017virus}, but many viruses remain named by their host, location, or symptoms.
This naming scheme may cause confusion as it results in many distinct viruses having the same name.
The Baltimore classification system standardizes viral classification by intrinsic morphological characteristics of a virus' replication machinery. 
It has been integrated into the ICTV guidelines to better reflect viral evolutionary relationships \citep{international2020new}.
The study of plant viruses particularly suffers from poor nomenclature due to naming a virus after a first discovered host which is subject to reclassification or renaming. %rephrase
The term ``plant virus'' in itself is problematic since there is strong evidence to suggest that viruses frequently spillover from fungal or invertebrate hosts \citep{lefeuvre_evolution_2019}.
Additionally, many plant viruses that infect agriculturally important species are named using the common name of a plant, which carries its own problems, for example: \textit{Pitaya Virus X} is named for the common name ``Pitaya'' which can refer to as many as thirty-one species within the genus \textit{Selenicereus} \citep{korotkova_phylogenetic_2017,guerrero_phylogenetic_2019,le_bellec_12_2011}. %rephrase, less passive voice
The matter is further complicated because one virus may infect many hosts, and one host may contain many viruses. 
A single-stranded RNA virus has a faster rate of evolution than a host plant.
It engages in a different mode of reproduction, making a direct assignment of viruses and their hosts difficult (citation needed).%rephrase
There is no guarantee that viral evolution and speciation follow linearly behind plant evolution and speciation---especially due to viral host-switching.
These problems persist throughout the genus \textit{Potexvirus} and are especially prominent in cactus-infecting \textit{Potexvirus} species.
We suggest a phylogeny-based approach to remedy some prominent taxonomic issues within this specific clade that cause naming inconsistencies.

%B: Hmmm, Intro is usually not allowed to have subsections. Also, Study System, Study Organism, or Study Site is commonly a subsection of Materials and Methods? Maybe, for reference, take a look at the organization of another paper? Karolis's paper is a good starting point, but any paper from a journal where you intend to send the ms is probably good, too.
%at: subsections removed.

% This subsection text, if in the M&M section should only be about the material that we are contributing. 
% If it is intended as background/review, those parts ought to be in the Introduction.
%at: I will trim this down soon and make it introduction-worthy.

The species \textit{Cactus Virus X}, \textit{Zygocactus Virus X}, \textit{Schlumbergera Virus X}, \textit{Pitaya Virus X}, and \textit{Opuntia Virus X} are all \textit{Potexviruses} (family Alphaflexiviridae) that are grouped broadly by their infections of certain cacti: \textit{Selenicereus undatus} and \textit{S. polyrhizus} \citep{li_viral_2015,peng_molecular_2016}; \textit{Opuntia spp.} especially \textit{O. tuna} \citep{koenig_molecular_2004, duarte_Potexvirus_2008} and \textit{O. monacantha} \citep{attathom_occurrence_1978} Sammons 1961 Duarte 2008; \textit{Schlumbergera} (previously \textit{Zygocactus}) \textit{truncata} and \textit{S. bridgesii} \citep{duarte_Potexvirus_2008, koenig_molecular_2004}, \textit{Parodia }(previously \textit{Notocactus}) \textit{leninghausii} \citep{park_detection_2018}, \textit{Echinopsis chamaecereus f. cristata}, \textit{E. pectinatus f. cristata}, \textit{E. jusbertii}, and \textit{E. macrogona} \citep{maliarenko_cactus_2013}; \textit{Mammillaria elongata f. cristata} \citep{maliarenko_cactus_2013}; and multiple other species within many genera in the family Cactaceae \citep{evallo_brief_2021}. Of these viruses, only \textit{Cactus Virus X} (CVX) has been reported on wild \textit{Ferocactus cylindraceus} (previously \textit{Ferrocactus acanthodes}) \citep{attathom_occurrence_1978}. 
However, this report predates DNA records confirming the viral identity.
Additionally, the viruses are frequently manipulated with serological experiments and have been found to produce lesions (which indicate infection) on: \textit{Chenopodium murale L.} \citep{maliarenko_cactus_2013} and C. quinoa \citep{attathom_identification_1978,attathom_occurrence_1978, brandes_untersuchungen_1963-1}; Nicotiana alata Link el. Otto \citep{maliarenko_cactus_2013}; Four species of Amaranthaceae \citep{attathom_identification_1978}; Escobaria vivipara \citep{attathom_identification_1978}; and other Cactaceae \citep{attathom_identification_1978}.

All cactus-infecting \textit{Potexviruses} consist of roughly 6,600 bp of positive-sense single-stranded RNA. They have similar rod-shaped filamentous virions and share the same division of five primary open reading frames (ORFs): Replicase (Rep), Triple gene block (TGB), Coat protein (CP), coded in the 5' direction as well as two smaller overlapping ORFs coded in the 3' direction: ORF6 and ORF7 \citep{evallo_brief_2021,liou_complete_2004, martelli_family_2007}.
They are closely related to other \textit{Potexviruses} such as \textit{Alternantha Mosaic Virus} and \textit{Papaya Mosaic Virus} \citep{martelli_family_2007,park_detection_2018,liou_complete_2004}.
These viruses produce a wide range of symptomatic and damaging infections in cacti. 
Reports of symptomatic plants range from 5.5 percent of wild \textit{Ferocactus cylindraceus} \citep{attathom_occurrence_1978} and up to 44 percent of crop plants on Hainan Island, China \citep{peng_molecular_2016}.
However, many infected plants do not show external signs of viral infection \citep{liou_complete_2004, bos_symptoms_1977}.
The most commonly recognized symptoms of the disease are mosaic, mottling, stunted growth and distortion \citep{maliarenko_cactus_2013, peng_molecular_2016, attathom_occurrence_1978}.

It is unclear what the method of transmission from infected plant to new host is. 
Some reports specify that cactus-infecting \textit{Potexviruses} can only be transmitted through grafting \citep{duarte_Potexvirus_2008, martelli_family_2007} but most agree that transmission can occur through mechanical contact such as sap inoculation \citep{liou_complete_2004,maliarenko_cactus_2013,park_detection_2018} and external tissue contact.
Grafting is a primary means of propagation among crop cacti \citep{park_detection_2018}, and \textit{Selenicereus} is a commonly chosen graft stock.
However, there are reports of other members within the family \textit{Alphaflexiviridae} transmitting via insect and seed vectors \citep{martelli_family_2007}, and pre-DNA studies tentatively suggest that in the wild, pollen may transmit \textit{CVX} \citep{attathom_occurrence_1978}.

Knowledge about cactus-infecting \textit{Potexviruses} contributes to a growing yet biased study of plant viruses. 
Human-assisted dispersal, grafting, and cultivation obscures the evolutionary history of these viruses, which parallels the disproportionate sampling representation of plants raised in greenhouses or for agricultural production. 
However, \textit{Cactus Virus X} and associated viruses seem restricted to cactaceous hosts for unknown reasons---every sample of CVX or CVX-related viruses has come from cacti.
The few studies that have investigated wild \textit{Potexviruses} of cacti predate DNA methods and have yet to identify the origin.
Recent sequencing efforts have revealed multiple inconsistent virus-host pairs on cacti.
Although many metagenomic studies capture environmental, genetic information that allows for virus identification, tissue type may bias expression rates of viruses \citep{lacroix2016methodological}.
The pursuit of wild cactus-infecting \textit{Potexviruses} expands our evolutionary knowledge of viral evolution, host selection, and transmission mechanics. 
The relationships of the virus can be investigated with a thorough phylogenetic approach, using available virus samples. 
In this study we present the largest to date phylogeny of cactus-infecting \textit{Potexviruses}.
We attempt to use this expanded phylogeny to answer relevant questions about Potexvirus evolutionary relationships and revisit the utility of decades-old taxonomy in current virus research. 


% METHODS ====================================================================
%------------------------------------------------------------------------------
\section*{Materials and Methods}
%------------------------------------------------------------------------------

\subsection*{RNA Sequencing}

Pistils (without ovaries), pollen, leaf, and root tissues were collected and immediately submerged in 1.5 ml of RNA\textit{later}\texttrademark~solution (Invitrogen).
Submerged samples were held at room temperature for thirty minutes and then moved to a -80 C freezer for storage.
Approximately 100 mg of tissue was ground to a fine powder in 1.5 ml tubes submerged in liquid nitrogen.
Total RNA was isolated using Total RNA Mini Kit (Plant kit; IBI Scientific, Cat. No. IB47341) following manufacturer's instructions.
We assessed RNA concentration and purity with a NanoDrop\texttrademark~Lite Spectrophotometer (Thermo Scientific).
The XX samples used in this study were sequenced as part of a larger sequencing effort which consisted of XXX separate sequencing runs and included additional samples from other plant species.
Sequencing libraries were prepared using the KAPA Stranded mRNA-Seq (Roche)
These libraries were sequenced on a single lane of Illumina \mbox{HiSeq}~4000 or Illumina \mbox{NovaSeq}~6000 platform (paired-end 150 bp reads) at the Duke University Center for Genomic and Computational Biology.
The number of resulting read pairs (for the XX samples presented here) ranged from X,XXX,XXX to X,XXX,XXX with a median of X,XXX,XXX and average of X,XXX,XXX (Table~S1).

\subsection*{RNAseq Assemblies}

Raw paired-end Illumina reads were first processed using \mbox{Rcorrector}~v1.0.4 
%\cite{song2015}
 to correct for random sequencing errors.
Then, reads were trimmed with \mbox{Trimmomatic}~v0.39 
%\cite{bolger2014}
 to remove any read containing bases with Phred scores lower than 20, low quality reads less than 50 bp long, and any adapter or other Illumina-specific sequences that were still present.
The remaining reads were filtered with \mbox{Kraken}~2 
%\cite{wood2019}
 to remove Small and Large Subunit ribosomal RNA (SILVA database) 
 %\cite{quast2013} 
 and contaminating reads (minikraken2\_v2 database).
We used custom-built databases, derived from RefSeq libraries: UniVec\_Core, viral, mitochondrion, plastid, plasmid, archaea, bacteria, protozoa, human, and fungi to minimize the number of contaminating and non-nuclear reads.
Only paired reads were used for transcriptome assemblies.
\textit{Schlumbergera truncata} filtered reads were combined across all samples into a single RNA-seq data set.
We conducted a \textit{de novo} transcriptome assembly using \mbox{Trinity}~v2.8.5 
%\cite{grabherr2011} 
to generate a single reference transcriptome assembly for \textit{Schlumbergera truncata}.

\subsection*{NCBI Data Collection and Compilation}
We collected publicly available genomes, complete proteins, gene annotations, and available metadata from Potexviruses (NCBI:txid12176) (NCBI: www.ncbi.nlm.nih.gov/, accession numbers provided in Supplemental Data). 
The untranslated regions (UTRs) were trimmed from the sequences to provide consistency.


We also searched the NCBI Sequence Read Archive (SRA) database (www.ncbi.nlm.nih.gov/sra) for RNA-sequencing (RNA-seq) data within the flowering plant order Caryophyllales (NCBI:txid3524) that had been sequenced using the Illumina library sequencing platform. 
For each identified SRA run accession (SRR), viral RNA that matched sample cactus-infecting Potexvirus RNA (accession numbers provided in Supplemental Data) was identified, extracted, and assembled using the kakapo 0.7.3-dev pipeline (http://flightless.one) with Kraken2 viral filters disabled. 
The SAM files produced through kakapo were loaded through Geneious 11.1.5 along with the Schlumbergera reads.
%KR check
These sequences were annotated using the Geneious 11.1.5 "Find ORFs" function. 

The complete dataset comprises 37 existing Potexvirus genomes and proteins, four new viral sequences located within original Schlumbergera truncata RNA-seq data, and 52 viral sequences found within NCBI Caryophyllales RNA-seq data.

\subsection*{Sequence Alignment and Phylogenetic Analyses}
Sequence alignments were performed through MAFFT v7.429 (\citep{katoh_mafft_2002}) using the full dataset.
%at: KR, can you check this? which MAFFT settings?
The aligned sequences were divided by ORF using the annotations to produce five partial sequence alignments corresponding to each ORF to accompany the full-sequence alignment. 
The individual proteins were exported to FASTA files, then gaps at the start of the sequence and stop codons were removed manually. 
Phylogenetic relationships and bootstrap values were calculated using IQtree v1.6.12 (\citep{nguyen_iq-tree_2015}), ModelFinder (\citep{kalyaanamoorthy_modelfinder_2017}), and UFBoot (\citep{hoang_ufboot2_2018}) for both the individual gene/protein alignments and the full sequence alignment. 
Trees were visualized in R version 4.0.3 using ggtree v2.4.2 (\citep{yu_ggtree_2017}). 
Host information was obtained through reported metadata and mapped onto the phylogeny. 
Species groupings were determined using the existing species boundaries compared to the phylogenetic branch lengths within the Potexvirus genus. 
These groupings were generally consistent with most recent branch lengths over 0.1 subs/site and this value was therefore used as a cutoff. 
Pairwise distance analysis was conducted on the sequence alignments in R using the ape v5.5  dist.dna() function with a raw model. 
For each defined clade, nonzero pairwise distances between each possible combination of tips was averaged. 
Expanded phylogenetic trees and individual gene/protein trees are available in the Supplementary Data.




%------------------------------------------------------------------------------
\section*{Results and Discussion}
%------------------------------------------------------------------------------
\subsection*{Characterization}
The collection of \textit{Schlumbergera} samples and thorough investigation of previously published data on Cactaceae resulted in XX new virus lineages. (Figure 1)
The genome sizes of X,XXX - X,XXX bp were consistent with published genomic \textit{Potexvirus} data, ranging from X.Xk - X.Xk bp.
The XXX sample reads recovered XX percent of the \textit{CVX} genome for the newly discovered viruses with \textit{Selenicereus} hosts. 
All of the publicly available new viral lineages were found on \textit{Selenicereus} hosts from SRA XXXXX.
We annotated the open reading frames of the viruses to recover all seven Potexvirus proteins.

The \textit{Schlumbergera}-infecting viruses were found in high amounts on pollen and style tissue.
The viral loads of each \textit{Selenicereus} sample that was found to have viruses ranged from XX-XX percent of all reads.
This was a relatively high recovery rate and in some pollen tissues there existed more viral reads (XX,XXX reads) than \textit{Schlumbergera} reads (XX,XXX reads).

\subsection*{Distribution of Genetic Distances}
The highly similar and well-clustered newly discovered viruses displayed low diversity within the clusters.
Therefore, these additions to the \textit{Potexvirus} family tree do not drastically alter the tree structure.
Since each sampled cactus in SRA XXXXX was located close to other sampled cacti, this low diversity potentially represents the first example of background mutation among viruses incurred due to host infection.
The average nonzero pairwise distance between the included subset of related \textit{Potexviruses} was 0.256 (maximum = 0.492). 
When the outgroup (including \textit{Plantago asiatica MV, Alternantha MV, Papaya MV,} etc.) was excluded from pairwise analysis, the average nonzero pairwise distance value was 0.177 (maximum = 0.326). 
When these cactus-infecting \textit{Potexviruses} were subdivided into six groups of relatively recent diversification, the average nonzero pairwise distance for full-genome sequences among groups was always above 0.015. 
The newly discovered \textit{Schlumbergera}-infecting viruses displayed XX percent similarity, and the \textit{Selenicereus}-infecting viruses from existing cactus samples showed XX percent similarity.
For the genes RNA-dependent RNA polymerase (RdRp) and Coat protein (CP), which the ICTV recommends be analyzed for species delimitation, the average nonzero pairwise distance was always above 0.02 (Figure 2). 
This correlates to roughly greater than 97.5 and 98 nucleotide identity. 


The ICTV guidelines for \textit{Potexviruses} indicate that less than 72 percent nucleotide sequence identity (or 80 percent amino acid identity) between the CP or Rep genes demarcates separate viral species. %clarify 
Because we compare closely related \textit{Potexviruses}, it might be expected that members of the same putative species would have higher than 72 percent nucleotide identity and members of different putative species would have lower than 72 percent sequence identity.
However, the low pairwise distances between \textit{Potexviruses} cause very few cactus-infecting \textit{Potexviruses} to be demarcated as separate species, even when only considering previously described species compared to each other.
Examples here.


\subsection*{Phylogenetic Relationships}
A well-supported phylogenetic tree was recovered including closely related non-cactus-infecting \textit{Potexviruses}.
The phylogenetic tree (Figure 1) places the new viral sequences from \textit{Schlumbergera} and \textit{Selenicereus} near existing viral species within \textit{Potexvirus} (Figure 1). 
Phylogenetic analysis recovered defined monophyletic groups corresponding to five or six major groups of cactus-infecting \textit{Potexviruses}, with \textit{Cactus Virus X} displaying two branching subgroups.
\textit{Opuntia Virus X} was the sister clade to the rest of the cactus-infecting \textit{Potexviruses}.
The \textit{S. truncata }samples located within the \textit{Cactus Virus X }clade appear to represent the first known discovery of \textit{Cactus Virus X} on \textit{Schlumbergera}.

Reported host genera are presented for each viral sample. 
The reported taxonomy of each existing sample aligns well with the tree structure, but this monophyly is not recovered for hosts.
Putative viral species appear to infect as few as one genus, in the case of \textit{Opuntia Virus X}, or as many as three in the case of \textit{Schlumbergera Virus X}.
Further, the three genera are evolutionarily distinct and polyphyletic in phylogenetic analyses of \textit{Cactaceae}. 
This relationship raises questions about a \textit{Potexvirus}'s ability to switch hosts. %clarify
All three genera that are reported hosts of \textit{Schlumbergera Virus X} are ornamental crops: \textit{Selenicereus, Schlumbergera}, and \textit{Opuntia}.
Extended greenhouse contact between the three genera may have resulted in viral spillover of \textit{Schlumbergera Virus X}.
However, it is also important to note that the present viral taxonomy is not infallible. 
The putative species known as \textit{Schlumbergera Virus X} may have been initially - incidentally - found on a spillover host rather than a member of an actual viral circulating population.
Unfortunately there is no way to know the "original host" of any virus that possesses the mechanisms to infect multiple cacti genera. 
Further testing of known host genera is necessary, and metagenomic sequencing of closely related genera to monitor potential opportunistic spillover. 

Taxonomic descriptions must also be analyzed for accuracy and reflection of actual viral activity. 
A virus named for its first known host may not represent the evolutionary history of the virus.
Although phylogenetic analysis is likely to produce monophyletic clades of viruses that have each been named in succession for the first known member, this produces inconsistent and confusing names.
\textit{Zygocactus Virus X} is a clear example: Although the name \textit{Zygocactus} is outdated in reference to the host genera, now classified as \textit{Schlumbergera}, the viral name remains.
This is exacerbated by the presence of a separate viral species, \textit{Schlumbergera Virus X}.
The ICTV strongly opposes unnecessary name changes, but it is unclear what should be done with outdated and confusing names that describe an extant clade of viruses.
Mixed infections and the lack of a one-to-one correlation between virus and host complicate naming endeavors further, and more sampling will uncover novel hosts and mixed infections.

\subsection*{Recombination and selection analysis}
Recombination events, how they were detected.
Selection analysis goes here.

\subsection*{Grouping}
The new samples fall into already existing species groups.
Pairwise distances between species were calculated for six groups, with a value of 0 indicating identical sequences and a value of 1 indicating entirely dissimilar sequences.

\subsection*{Concluding Remarks}
X new viruses were included as part of this study, which represents a multifaceted approach to viral discovery using metagenomic techniques for already available public data as well as newly collected data.
Mixed infections, co-infection dynamics.
Viral abundance, viral loads, tissue type, and diversity.
Transmission: Sexual
Taxonomic recommendations - the current species names should not be used in the future.
%------------------------------------------------------------------------------
\section*{Acknowledgments}
%------------------------------------------------------------------------------

Acknowledgments text.

% BIBLIOGRAPHY ================================================================
\pagebreak
\raggedright{}
\setstretch{1.0}
\setlength{\parindent}{0.0in}
\bibliographystyle{evolution}
%{\fontsize{10pt}{15pt}\selectfont}
\bibliography{cvx-refs}
%}
\end{linenumbers}
% FIGURES =====================================================================
\pagebreak
\frenchspacing
\setstretch{1.0}
\setlength{\parindent}{0.0in}

\section*{Figures}
Figure inventory list.
1. The updated cactus-infecting \textit{potexvirus} phylogeny reflects distinct viral groupings which opportunistically infect multiple host plants. Asterisks (*) represent samples recovered by this study. 

2. Genome organization of viruses in phylogeny, ORFs color-coded and arrows indicating which direction each gene is read.
3. Selection graph as displayed in B's r code
4. Table of pairwise distances in nucleotides
5. Graph of pairwise distances
6. Proposed update to potexvirus taxonomy.

Supplemental figures:
1. full gene phylogenies for each gene in genome
2. Tissue expression rates?



% \begin{figure}[ht]
% \centering
% \includegraphics[width=0.5\linewidth]{FILE_PATH}
% \begin{NoHyper}
% \caption{{\fontsize{10pt}{11pt}\selectfont}
% \label{fig:FIG_LABEL}
% \end{NoHyper}
% \end{figure}

% END =========================================================================

\end{document}
