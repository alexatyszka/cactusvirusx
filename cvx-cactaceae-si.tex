% -*- program: pdflatex -*-

\documentclass[11pt]{article}
\usepackage{evolution}
\usepackage[section]{placeins}

\begin{document}

\begin{center}
	\includegraphics[width=0.4\textwidth]{newphyt/new-phytologist-si-logo.jpg}
\end{center}

\section*{New Phytologist Supporting Information: Figures S1-S5 and Tables S1-S4}
\bigskip
\textsc{Article title}: \textbf{\em RNase-based Self-incompatibility in Cacti}\\
\textsc{Authors}: \textbf{\em Karolis Ramanauskas \& Boris Igi{\'c}}\\
\textsc{Article acceptance date}:  \textbf{\em28 May 2021}.

\clearpage

%
% FIGURES
%
%Fig S1: Gene tree of T2/S-RNases
\begin{suppfigure}
\centering
\caption{Gene tree of T2/S-RNase gene family.
The phylogenetic relationships were inferred from a representative sample of three described ``classes'' of eudicot T2/S-type RNases (see Ramanauskas \& Igi{\'c} 2017).
Tip labels list species binomials, gene names (if given), and GenBank accession IDs.
Circles placed in each node show percentage bootstrap support for subtending branches, out of 1000 resampling runs.
White nodes receive bootstrap values $<50$\%, green nodes $\geq50$\%.
The RAxML reported log-likelihood score was $-56941.297900$ (v8.2.12, model \texttt{-m GTRGAMMA}).
S-RNase sequences from the families with RNase-based SI are colored and marked.
The scale shown illustrates the vast divergences, in units of nucleotide substitutions per site.
Many pairwise distances within each class exceed one substitution per site, and many distances between classes exceed two substitutions per site, limiting precision of inference.
Nevertheless, with the exception of presumed non-functional or neofunctionalized paralogs (in black), S-RNases form a monophyletic group within Class III T2/S-type RNases.
}
\includegraphics[width=0.404\textwidth]{figures/tr_rns_ve}
\label{fig:rnase-tree}
\end{suppfigure}
\clearpage


%
% TABLES
%

\begin{supptable}[ht]
\centering
\caption{RNA-seq and assembly summary statistics.
The length of all reads before trimming was 150 bp.
Column labelled \textsuperscript{*}filt. lists numbers of read pairs and average qualities after filtering step.} %FIXME: missing most of the caption.
% To browse your data, use SRA RunSelector:
% https://www.ncbi.nlm.nih.gov/Traces/study/?acc=PRJNA705387
% To see your summary list view (availability at this link subject to 24 hour delay):
% https://www.ncbi.nlm.nih.gov/sra/?term=PRJNA705387
\label{tab:suppSeqAsmbStats}
\resizebox{\textwidth}{!}{
\begin{tabular}{l|cc|cc|cc|cc|cc|c}
\toprule
\multicolumn{1}{c|}{} &
\multicolumn{2}{c|}{Read~pairs} &
\multicolumn{2}{c|}{Avg.~read~len.~trimmed} &
\multicolumn{2}{c|}{Avg.~qual.} &
\multicolumn{2}{c|}{Avg.~qual.~\textsuperscript{*}filt.} &
\multicolumn{2}{c|}{Assembly} &
\multicolumn{1}{c}{} \\
\multicolumn{1}{c|}{Sample} & Raw & \textsuperscript{*}filt. & F & R & F & R & F & R & Isoforms & Genes & SRR \\
\midrule
\textit{Schlumbergera truncata} 15H01 pol   & 4,841,466 & 3,353,072 & 149 & 147 & 35.8 & 34.8 & 36.2 & 35.8 & 43,294 & 27,125 & \href{https://trace.ncbi.nlm.nih.gov/Traces/sra/?run=SRR13805650}{SRR13805650} \\
\textit{Schlumbergera truncata} 15H01 sty   & 6,769,308 & 4,628,401 & 148 & 139 & 38.8 & 35.9 & 39.9 & 38.8 & 73,047 & 42,717 & \href{https://trace.ncbi.nlm.nih.gov/Traces/sra/?run=SRR13805653}{SRR13805653} \\
\textit{Schlumbergera truncata} 15H02 pol   & 4,888,079 & 3,353,774 & 149 & 147 & 35.8 & 35.0 & 36.2 & 35.8 & 48,134 & 30,138 & \href{https://trace.ncbi.nlm.nih.gov/Traces/sra/?run=SRR13805649}{SRR13805649} \\
\textit{Schlumbergera truncata} 15H02 sty   & 6,173,982 & 4,263,304 & 147 & 139 & 38.8 & 36.3 & 39.8 & 38.8 & 56,416 & 32,918 & \href{https://trace.ncbi.nlm.nih.gov/Traces/sra/?run=SRR13805652}{SRR13805652} \\
\textit{Schlumbergera truncata} 15H03 pol   & 6,983,951 & 4,858,975 & 149 & 147 & 35.8 & 35.1 & 36.2 & 35.9 & 53,675 & 32,640 & \href{https://trace.ncbi.nlm.nih.gov/Traces/sra/?run=SRR13805648}{SRR13805648} \\
\textit{Schlumbergera truncata} 15H03 sty 1 & 9,183,100 & 5,450,149 & 147 & 139 & 38.4 & 35.9 & 39.6 & 38.6 & 59,668 & 34,275 & \href{https://trace.ncbi.nlm.nih.gov/Traces/sra/?run=SRR13805641}{SRR13805641} \\
\textit{Schlumbergera truncata} 15H03 sty 2 & 7,147,068 & 4,815,235 & 148 & 139 & 38.7 & 36.1 & 39.8 & 38.7 & 48,708 & 28,407 & \href{https://trace.ncbi.nlm.nih.gov/Traces/sra/?run=SRR13805637}{SRR13805637} \\
\textit{Schlumbergera truncata} 15H04 pol   & 8,392,597 & 5,497,308 & 148 & 138 & 38.8 & 35.8 & 39.8 & 38.6 & 58,662 & 35,689 & \href{https://trace.ncbi.nlm.nih.gov/Traces/sra/?run=SRR13805647}{SRR13805647} \\
\textit{Schlumbergera truncata} 15H04 sty   & 5,240,352 & 3,446,465 & 147 & 140 & 38.6 & 36.4 & 39.8 & 38.8 & 54,120 & 31,872 & \href{https://trace.ncbi.nlm.nih.gov/Traces/sra/?run=SRR13805636}{SRR13805636} \\
\textit{Schlumbergera truncata} 15H05 pol   & 6,347,070 & 4,365,031 & 148 & 139 & 38.8 & 35.9 & 39.8 & 38.8 & 46,638 & 28,580 & \href{https://trace.ncbi.nlm.nih.gov/Traces/sra/?run=SRR13805646}{SRR13805646} \\
\textit{Schlumbergera truncata} 15H05 sty 1 & 9,618,084 & 6,479,817 & 146 & 139 & 38.2 & 35.9 & 39.6 & 38.8 & 71,031 & 41,443 & \href{https://trace.ncbi.nlm.nih.gov/Traces/sra/?run=SRR13805635}{SRR13805635} \\
\textit{Schlumbergera truncata} 15H05 sty 2 & 5,043,649 & 3,051,699 & 147 & 141 & 38.7 & 36.8 & 39.9 & 39.0 & 59,857 & 36,264 & \href{https://trace.ncbi.nlm.nih.gov/Traces/sra/?run=SRR13805634}{SRR13805634} \\
\textit{Schlumbergera truncata} 15H06 pol   & 6,850,087 & 4,360,475 & 148 & 141 & 38.9 & 36.5 & 39.7 & 38.8 & 27,729 & 18,571 & \href{https://trace.ncbi.nlm.nih.gov/Traces/sra/?run=SRR13805645}{SRR13805645} \\
\textit{Schlumbergera truncata} 15H06 sty   & 6,632,382 & 4,203,598 & 148 & 139 & 38.8 & 36.1 & 39.7 & 38.6 & 24,392 & 16,949 & \href{https://trace.ncbi.nlm.nih.gov/Traces/sra/?run=SRR13805633}{SRR13805633} \\
\textit{Schlumbergera truncata} 15H07 pol   & 4,513,581 & 2,988,451 & 148 & 147 & 35.7 & 35.0 & 36.2 & 35.9 & 46,908 & 29,464 & \href{https://trace.ncbi.nlm.nih.gov/Traces/sra/?run=SRR13805644}{SRR13805644} \\
\textit{Schlumbergera truncata} 15H07 sty   & 6,411,850 & 4,200,344 & 148 & 139 & 38.9 & 35.9 & 39.9 & 38.6 & 68,107 & 39,657 & \href{https://trace.ncbi.nlm.nih.gov/Traces/sra/?run=SRR13805632}{SRR13805632} \\
\textit{Schlumbergera truncata} 15H08 pol   & 5,106,699 & 3,863,997 & 149 & 147 & 35.8 & 35.1 & 36.2 & 35.8 & 36,739 & 23,618 & \href{https://trace.ncbi.nlm.nih.gov/Traces/sra/?run=SRR13805643}{SRR13805643} \\
\textit{Schlumbergera truncata} 15H08 sty   & 4,521,102 & 2,741,695 & 148 & 146 & 35.7 & 34.8 & 36.2 & 35.8 & 47,754 & 29,812 & \href{https://trace.ncbi.nlm.nih.gov/Traces/sra/?run=SRR13805631}{SRR13805631} \\
\textit{Schlumbergera truncata} 15H09 pol   & 4,645,325 & 2,575,494 & 148 & 147 & 35.7 & 35.0 & 36.2 & 35.8 & 42,245 & 27,650 & \href{https://trace.ncbi.nlm.nih.gov/Traces/sra/?run=SRR13805642}{SRR13805642} \\
\textit{Schlumbergera truncata} 15H09 root  & 6,653,640 & 3,261,445 & 149 & 148 & 35.7 & 34.4 & 36.4 & 36.2 & 68,427 & 48,463 & \href{https://trace.ncbi.nlm.nih.gov/Traces/sra/?run=SRR13805640}{SRR13805640} \\
\textit{Schlumbergera truncata} 15H09 stem  & 8,275,870 & 5,731,805 & 149 & 149 & 35.8 & 35.2 & 36.4 & 36.2 & 79,859 & 51,788 & \href{https://trace.ncbi.nlm.nih.gov/Traces/sra/?run=SRR13805639}{SRR13805639} \\
\textit{Schlumbergera truncata} 15H09 sty   & 6,363,556 & 4,525,356 & 147 & 138 & 38.8 & 35.8 & 39.8 & 38.7 & 62,991 & 36,275 & \href{https://trace.ncbi.nlm.nih.gov/Traces/sra/?run=SRR13805651}{SRR13805651} \\
\textit{Matucana madisoniorum} HBG13 sty    & 4,148,932 & 2,396,890 & 145 & 140 & 38.1 & 36.4 & 39.5 & 38.9 & 41,968 & 33,810 & \href{https://trace.ncbi.nlm.nih.gov/Traces/sra/?run=SRR13805638}{SRR13805638} \\
\bottomrule
\end{tabular}}
\end{supptable}
\clearpage
\bigskip

\clearpage

\section*{References}

\textbf{Ramanauskas~K}, \textbf{Igi{\'c}~B}.
\newblock \textbf{2017}.
\newblock {The evolutionary history of plant T2/S-type ribonucleases}.
\newblock \emph{PeerJ} \textbf{5}: e3790.
\end{supptable}

\end{document}
